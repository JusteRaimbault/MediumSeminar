%%%%%%%%%%%%%%%%%%%%%%%%%%%%%
% Standard header for working papers
%
% WPHeader.tex
%
%%%%%%%%%%%%%%%%%%%%%%%%%%%%%

\documentclass[11pt]{article}

%%%%%%%%%%%%%%%%%%%%
%% Include general header where common packages are defined
%%%%%%%%%%%%%%%%%%%%



%%%%%%%%%%%%%%%%%%%%%%%%%%
%% TEMPLATES
%%%%%%%%%%%%%%%%%%%%%%%%%%


% Simple Tabular

%\begin{tabular}{ |c|c|c| } 
% \hline
% cell1 & cell2 & cell3 \\ 
% cell4 & cell5 & cell6 \\ 
% cell7 & cell8 & cell9 \\ 
% \hline
%\end{tabular}





%%%%%%%%%%%%%%%%%%%%%%%%%%
%% Packages
%%%%%%%%%%%%%%%%%%%%%%%%%%



% encoding 
\usepackage[utf8]{inputenc}
\usepackage[T1]{fontenc}


% general packages without options
\usepackage{amsmath,amssymb,amsthm,bbm}

% graphics
\usepackage{graphicx,transparent,eso-pic}

% text formatting
\usepackage[document]{ragged2e}
\usepackage{pagecolor,color}
%\usepackage{ulem}
\usepackage{soul}







%%%%%%%%%%%%%%%%%%%%%%%%%%
%% Maths environment
%%%%%%%%%%%%%%%%%%%%%%%%%%

%\newtheorem{theorem}{Theorem}[section]
%\newtheorem{lemma}[theorem]{Lemma}
%\newtheorem{proposition}[theorem]{Proposition}
%\newtheorem{corollary}[theorem]{Corollary}

%\newenvironment{proof}[1][Proof]{\begin{trivlist}
%\item[\hskip \labelsep {\bfseries #1}]}{\end{trivlist}}
%\newenvironment{definition}[1][Definition]{\begin{trivlist}
%\item[\hskip \labelsep {\bfseries #1}]}{\end{trivlist}}
%\newenvironment{example}[1][Example]{\begin{trivlist}
%\item[\hskip \labelsep {\bfseries #1}]}{\end{trivlist}}
%\newenvironment{remark}[1][Remark]{\begin{trivlist}
%\item[\hskip \labelsep {\bfseries #1}]}{\end{trivlist}}

%\newcommand{\qed}{\nobreak \ifvmode \relax \else
%      \ifdim\lastskip<1.5em \hskip-\lastskip
%      \hskip1.5em plus0em minus0.5em \fi \nobreak
%      \vrule height0.75em width0.5em depth0.25em\fi}



%%%%%%%%%%%%%%%%%%%%
%% Idem general commands
%%%%%%%%%%%%%%%%%%%%
%% Commands

\newcommand{\noun}[1]{\textsc{#1}}


%% Math

% Operators
\DeclareMathOperator{\Cov}{Cov}
\DeclareMathOperator{\Var}{Var}
\DeclareMathOperator{\E}{\mathbb{E}}
\DeclareMathOperator{\Proba}{\mathbb{P}}

\newcommand{\Covb}[2]{\ensuremath{\Cov\!\left[#1,#2\right]}}
\newcommand{\Eb}[1]{\ensuremath{\E\!\left[#1\right]}}
\newcommand{\Pb}[1]{\ensuremath{\Proba\!\left[#1\right]}}
\newcommand{\Varb}[1]{\ensuremath{\Var\!\left[#1\right]}}

% norm
\newcommand{\norm}[1]{\left\lVert #1 \right\rVert}



% argmin
\DeclareMathOperator*{\argmin}{\arg\!\min}


% amsthm environments
\newtheorem{definition}{Definition}
\newtheorem{proposition}{Proposition}
\newtheorem{assumption}{Assumption}

%% graphics

% renew graphics command for relative path providment only ?
%\renewcommand{\includegraphics[]{}}





\renewcommand{\abstractname}{}




% geometry
\usepackage[margin=2cm]{geometry}

% layout : use fancyhdr package
\usepackage{fancyhdr}
\pagestyle{fancy}

\makeatletter

\renewcommand{\headrulewidth}{0.4pt}
\renewcommand{\footrulewidth}{0.4pt}
\fancyhead[RO,RE]{\textit{Seminar Report}}
\fancyhead[LO,LE]{Medium Project}
\fancyfoot[RO,RE] {\thepage}
\fancyfoot[LO,LE] {}
\fancyfoot[CO,CE] {}

\makeatother


%%%%%%%%%%%%%%%%%%%%%
%% Begin doc
%%%%%%%%%%%%%%%%%%%%%

\begin{document}









\title{MEDIUM\\\medskip
\textit{New Pathways for Sustainable Urban Development in China’s medium-sized cities}\\\medskip
Zhuhai Scientific Seminar - Report}
\author{}
\date{December 3-5th 2016}


\maketitle

\justify


\begin{abstract}
\end{abstract}






\section*{Day 1}

The first session of the day is chaired by Irene Poli (UNIVE).

\subsection*{Introduction}


%Welcome from SYSU. founding of SYSU. ranked top 10 in China, 1 in Guangdong.
%four depts and 8 research institues.
%human geo. national key discipline nationwide, as GIS/Remote sensing.
%Invites collaboration/cooperation.

\paragraph{Pr. Suhong Zhou greats welcomes on the behalf of SYSU.}

Brief presentation of SYSU (Departments, Research Institutes). At School of Geography, Human Geography and GIS/Remote Sensing as key disciplines.


\paragraph{Pr. Chunshan Zhou}

Welcome on behalf of SYSU.
Zhuhai typical medium sized city : feature and typical issues.
Economically developped area : representative case study.
Output of seminar $\rightarrow$ ref for dev of medium sized cities.

\paragraph{Pr. Natacha Aveline (CNRS) presents the Medium project.}

%Presentation of Medium. Research aims at sustainable dev in China. 

It aims at sustainable urban development in China, following three axis: Case study of three medium-sized cities; collaboration with three universities; young european researchers in China.

%3 topics : 
%\begin{itemize}
%\item Medium sized cities : 3 universities
%\item Collaboration with universities. Next year : Datong.
%\item More young researchers/experts from EU. Hangzhou/Zhuhai. Presents European universities.
%\end{itemize}

%Organisation UMR : CNRS / P1 / P7.
%(pres. CNRS).

%Thanks collegues ; Thanks Pr. Zhou he ta de xuesheng, especially CHenyi.






\subsection*{Pr{\'e}sentation of G{\'e}ographie-cit{\'e}s Research Unit}


(Martine : General Secretary)

%[Technical Problems]

Organisation : CNRS and P1 / P7. CNRS largest research center in Europe. all research fields.

59 researchers ; 20 ingeneers ; 70 phd students.

It has three teams : CRIA focusing on industry and planning ; EHGO : image in geography, developpement of knowledge ; PARIS : city and urban systems, spatial dynamics, science of complexity.

(Seb Haule) geomatics engineering. L Chalonge : cartography and Visualisation, statistical treatments. Can help anyone with doing maps/technical questions.
S. Haule : historian ; analysis of geohistorical data. website design, digital humanities. ERC Seasteems.

A. Banos : simulating complex spatial systems, new perspectives in geosimulation.


CNRS and China : late 70s (1978). 6 structures affiliated to inshs
Geocites particular relations : Medium since 2015 ; Finurbasie (Natacha's ANR) ; Odessa with Tsinghua and Sheffield ; ERC C Ducruet (7 CHinese researchers) ; Denise : Thèse Elfie.








\subsection*{The integration of medium-size Chinese cities in the globalized world: the profile integration of Zhuhai - Pr Celine Rozenblat}

join work with E Swerts and A Ignazzi.

\textbf{Research Question} Fast integration of Cihna in global economy

Assumption : small and medium size beneficiate lower skill industrial.
but could concentrate innovative firms.

-> how can Zhuhai become innovative.

Larger Q : Medium size cities.

Def ? by negation / relative to an urban system.

Def in China : [Henderson, Choi, Logan, 2005] : county level, administrative. ; Size : [Appleyard, Zheng et all, 2007] ; geographic position [Schneider, Mertes, 2014] : distance to big cities.

\textbf{Hypothesis : } Intensity of integration in the globalization.

Effects to test : admin , Urban Size, Geo position.
Specific economic profile of Zhuhai?

Multi-level cities networks (micro - meso - macro)

Data: Micro Networks Data (Bureau Van Dijk), 3000 Top world groups (Turnover), 8e5 subsidiaries (nodes) ; nodes : ownership.

aggregation : inter-urban network.

def of city : Large Urban Region (UNIL + ERG Geodivercity)

Elfie's database for China.

World network of cities (graph viz positioning).

In 2013 : change in Network topology (crisis).

continents are very cohesive. only 20\% are inter-continental.

China in 2010 : 61\% on inter-continential linkages ; intra-urban only 6\%
in 2013 : 10\% ; 45\% only inter-cont. 

Zoom on CHina. ! importance of Bahamas.
network more balanced.

middle size cities took advantage.

rank-size in ownership and subsidiaries (in and out degree) : exchange Hong-Kong/Beijing.

Middle size cities does a plateau in middle ? (middle size for company)

Scaling laws.

Test of predictive power of pop on network ? R2 =  2/3 rougly.
Change in regime, more importance of specialization ?

growth rates.

(Orbis db ?)

Zoom on 3 cities :
\begin{enumerate}
\item Hangzhou : Fund mgt activities/services mostly. Example of Zhejiang expressway company. Very diversified, thanks to Shangai proximity.
\item Datong : strong specialization, difficulty for attraction and diversification.
\item Zhuhai : 2nd port after Shenzen, airport, train bridge. SEZ since 80. National HiTech dev zone ; Free trade zone. Administrative measures : reduction of taxes. Example : Gree Eletrical Appliances. Strategic link with universities. Zhuhai quite well integrated, also in emerging countries. 2nd ex. : Livzon Pharmaceutical Group ; intensive collaboration with SYSU school of medecine and pharmacology (oldest of western medecine) ; mainly for CHinese market. Zhuhai has related and unrelated specializations, importance of universities ?
\end{enumerate}

COnclusion :
def of medium size by integration in globalization.
urban size effect decreasing.
administrative effect : spatial spread of emergent attractive cities
cities around metropolis.
Specialization : local ressources (Datong)

Specialization of Zhuhai as innovative place.
Plycentrism perspectivsm (proximity financial poles).

(recall forthcoming IGU Brasil 2017 ; Quebec 2018)



\textbf{Q Natacha} : Datong : involved in coal industry - strategic domain as coal (for central government). pb : no other companies.

\textbf{Q Suhong Zhou} City regions emerging - difficulty to define the city. -> Work now to define functional urban regions. Different levels. Functional urban region : same airport for international : accessibility to the world. (q : when many airports ?)

\textbf{Q Pr Hangzhou} Alibab very international ; why these particular companies ? $\rightarrow$ pb with alibab, settled in a fiscal heaven, one link only, no other company complexifies. should look at other companies.








\textit{Q : possible to have same database openly ?}







\subsection*{New pathways for sustainable urban development in Zhuhai}



Presentation of the Pearl River Delta : one of economic poles. largest urbanization rates ; largest GDP.

Very large population, plus floating labor (not in stats).

Urbanization in PRD :
\begin{itemize}
\item 80s. Transfer of industry ; Hong-Kong Macao ; Opening policy. Mode of development : bottom to top, development of villages ; infrastructure construction.
\item late 80s - mid 90 : construction of express ways and railways ; 92 : Deng Xiaoping strengthen opening policy.
\end{itemize}

one of highest density area. Spatial structure (cluster of cities).

Different level of development.

Modes : Dongguan (Foreign Investment) - East ; Shunde (private-owned, local investment industry cluster) - West ; Sehnzen (Special Economic zones : high-tech and service sector).


Division and Cooperation : Emergence of specialized clusters.

Situation of Zhuhai : lagging of eco dev. very low gdp but high urbanisation rate. Waving of economic development.80->95 : oscillating.

pb : investment efficiency is not high.

Resources : very good transportation, infrastructure, good tourism ressources. Good environmental protection. Also good real estate and living conditions.

advanced equipment manufacturing industry.

Zhuhai has a good spatial position in DPR : trhee triangles (Zhuhai-Shenzen-Guangzhou ; Zhuhai HK Macao ; Zhuhai Zhingsan ZIanmen).

Space-time trajectory : local development vs adjacent regions's development. 

Trajectories for 21 cities in Guangdong [Zhou Suhong, Gu Jie, Ye Xinyue, Yan Xiaoppei, 2015]. Cities that are in advances compared to regionalk average.

Importance of infrastructure dvlpmt.

Cooperation Zhuhai-HK-Macao : close, but not high level.
impacts : better integration of Zhuhai.
Maps : impact on accessibility. change on all structure on DPR (\textbf{RQ : we cant know that for sure (structural effects of transportation)})

Cl : rapid urbanization ; Special Economic Zone ; Lagging but high potential of dev ; New chances and New pathways.

\textbf{Q Irene Poli} Different profiles East/West ; different environmental politics ? $\rightarrow$ gap appeared at beginning of dev. now east upgrading. 


\textbf{Q Prof Venice} Sustainable Development indicators ? $\rightarrow$ not working directly. Q : city of Zhuhai working on it ?

\textbf{Q Prof Hangzhou} New opportunities brought by new bridge. Of course closely related to dev of HK/Macao. Difficult to break fixed pattern in SEZ. : don't need to break it.
-> on macro level, difficult to inverse it. Zhuhai capitalizing on this positioning ? good position in strategy of transformation of industries in China (new era). more potential to catch up with other ciies. 

\textbf{Q who paued for infrasctructure} : State goverment. : but also province (railways). [Zhuhai good infra (?)]





\paragraph{The Development of medium cities in China - Pr Zhou Chunshan}

Criteria for classyfying cities. by def, population < 1Mio. (permanent residents). Large, super-Large $\in [5e6,10e6]$ ; before was 20-50 for medium. def of population changed

Number and distribution by grades of cities. 2014 : 100 midlle size chengshi. 

At different dates : stable number of medium sized cities. Pattern : large and medium are evolving into big cities.
Distribution by regions : mainly located in Central China.

increase in rank by size in almost each province.
Map of spatial distribution.

Population : decreasing from 2005 to 2014. capacity to accomodate pop has decreased.

Average salary : lowest salary level : relatively poor, however rise in time, but also rise of costs.

Economy : fluctuation of GDP. Coastal areas and Beijing-Gauagngzhou railay. also lowest accross classif. (in per capita GDP). but rising. distribute in city clusters.
GDP growth rate : lowest gdp growth rate.

Fiscal revenue and expenditure : more expenditure than revenue.
tehcnology/education : lower on sci but high for education. Great spatial disparsity for science and education : northeast parts have lowest expenditures.
Financial index : deposits in financial instituations/loans

Industry : industrial structure. primary industry increase with decreasing size. NorthEast : more primary. 
Number of entreprises (over 5Mio RMB) : downward trend in middle size cities. by type : domestic, invested by HK Macao Taiwan, foreign.

Structure of jobs. expect large cities where secondary if most. tertiary most in middle, creating more jobs. map of jobs created by industry. primary : Beijing-Shangai ; secondary : Yangtze delta ; tertiary : middle, southern and West china.

Conclusion :
\begin{enumerate}
\item middle were majority, now large cities.
\item located in middle, central China
\item population high but decreasing trend ; gdp high, raising tren in per-capita gdp and average salary
\item but rate quite slow, low expednitures ; majority on education, lower on science.
\item industrial structure : secondary most
\item increasing foreign investment
\item property market contribute little
\item number of jobs created : secondary.
\end{enumerate}

Low efficiency in medium size cities ; same conclusion the other day in other seminars. migrants moving to medium now. advantages but need to improve efficiency.

\textbf{Q Suhan Zhou}  lowest income level, slow growth : possible reasons for lower pace ? -> surprsing result. medium cities have problem to increase housing prices. income : innovation and attracting investments: lower efficiency ; maybe large scale for investments, but not efficient. people going back to hometown after graduating.

\textbf{Q Liao Liao} criteria of classification : link with administrative classification. reason for low efficiency ? other countries to try to boost ? give more priority to small cities ? -> county level. cities with better efficiency : bigger than one million. policies : controlling the size, seeking rational devlopment. maybe need multi-dimensionnal approach. efficiency of dvlpmt and liveability of the city.




\subsection*{The Development strategy of Zhuhai city (Planning Bureau)}

Zhuhai core city in DPR, but lagged during 8 years.
characteristics and statistics. 2Mio real pop.

Awards : sustainable dvlpmt city ; most liveable city.

Spatial Strategies : different stages
\begin{enumerate}
\item Startup : 79-83 - agriculture and industry
\item Big economic dev 84-90 : bigger spatial dev zone.
\item 91-2000 : infrastructure, difficulty to raise funds.
\item 01-08 : devlpmt of west size
\end{enumerate}

88 : foreign oriented ; sez.
third phase : expansion to west.
98 : west corridor region of PRD. 4th version not very flexible.
Last version : 2015 ; next year : compare with new version. protecting natural ressources. cluster of econnomic zones, naturally formed (mountinas, water)

plan for 2060 : CBD in now wetland, plan ebated.

Problems : capacity of regional influence stays low. infrastriuctre congested, not completed, internal transportation not smooth. Zhuhai airport : small handling capacity. poor public facilities.

not ``efficient land use'' ? Imbalance between east and west.
townships upgraded but few public services. Lack on complete industry chain.

Future spatial strategies ? Guangdong HK Macao Bay area : create a world-level city cluster. all south bay area integrates innovation economic belt. (rq : third bridge at ecological island)
Zhuhai needs to increase radiation.
Region-oriented spatial development strategy. Bridge to increase interactions. Questionnaire for HK companies to invest in Zhuhai : radiation in West and facilities. exploring the west as a priority.
(...)
Land use patterns.
Future spaces for eco dev. agricutltural parks.
ecological patterns.
Traffic patterns : W/E - N/S
U shaped development belt.


Sustainable dev index system. regulation for buildings etc.
Pathways. Performance evaluation.
Index system of immediate plan.




 




\subsection*{The urban spatial structure towards low-carbon transportation and an exploration of low - carbon city construction}

21\% of CO2 emissions from transport sector.
increasing car ownership. Zhuhai is 28th on traffic congestion.

Way to low-carbon transport : 
\begin{itemize}
\item Low carbon energy
\item Traffic control
\item spatial structure
\end{itemize}


Form - Function - Network

Compact cities

Effective mixed land use [Pan Haixiao 2008]
[Gu and Zhou 2010]

Reference to foreign cases. Zurich as raw model for pedestrian cities : two-stage pedestrian network system.

TOD model. ex Singapore (1981 : new town planning).

Link building Complex / trnaposrt system.

Copenhaguen : finger-shaped ; also tod model. 

Europe-China low carbon eco-cities : Zhuhai and Luoyang. Zhuhai : green line planning. 

TOD townships planned in Zhuhai (ex north station)

low carbon transit + effective mixed land-use + liveable block size.

\textbf{Q (chairman)} air pollution monitoring system ?








\paragraph{A study on the construction strategy of Qi Ao ecological island in Zhuhai}

Major part of Zhuhai territory : islands.

mangrove area in north of island - very close to urban area : protected. beuatiful scenery. cultural ressources. 

policy support for the development of an eco-island. -> become a model ?

Mangrove : natural habitat for birds etc. eco-technologies ?

Threats : garbage, domestic rubish. clutivation : unregulated cultivation by fishermen, illegal fishing. Alien invasive species.

reduce, Reuse, Recycle. offshore wind powerplants.

Ecologial tourism (mangrove museum)

\textbf{Q : Pr Hangzhou} how many inhabitants ? size of the population, not develop island, consider 

\textbf{LiaoLiao} which level of administration ? tonwship or city govt -> plan for wetland parks.

\textbf{Irene Poli} which kind of political procedure for the pb of food for the population. 






\paragraph{A comprehensive evaluation on the development of industrial ecologicalisation and analysis of influencing factors - a case study of Zhuhai}


Industrial Ecologization : industries coordinates, industrial ecology.
Kalundborg, Denmark.
Guandong open economy.
interdisciplinary and integrated.

Zhuhai : mostly secondary industry.

Multiobjective : weighting.

(...)

\textbf{Q Natacha} bottom-up policies in China ? need to find more stimulus for that in Zhuhai.

\textbf{Q Valentina} beautiful environment : what ? (...)







\paragraph{Social integration of different types of migrant workers and analysis of the influencing factors: a case study of Shenzhen}






\paragraph{Social Spatial Structure in Zhuhai}






\section*{Day 2}

This morning session is chaired by .


\subsection*{Migrant workers and Chinese medium-size cities: the issue of migration in China’s new urbanisation strategy}

\textit{Cinzia Losavio, Université Paris 1, UMR CNRS 8504 Géographie-cités}




\subsection*{Towards a theory of co-evolutive networked territorial systems: insights from transportation governance modeling in Pearl river delta}

\textit{Juste Raimbault, Université Pars 7, UMR CNRS 8504 Géographie-cités}





\subsection*{Evolution of management in the development zone in Zhuhai: an emergence of a local model of governance}

\textit{Liao Liao, Sciences-Po Aix}



\subsection*{Agent-based modeling of migrant workers residential dynamics within a mega-city region: the case of Pearl river delta}

\textit{CL and JR}



\subsection*{Planning, representations and perceptions of China’s urbanization: a case study on Hangzhou Future Sci-Tech City}

\textit{Valentina Ansoize, UNIVE}





\subsection*{The reshaping of social groups through the process of urban renewal in post-socialist China: a case study on the coal miners in Datong}

\textit{Judith Audin, Sciences-Po Aix}





\paragraph{Conclusion}







\newpage

%%%%%%%%%%%%%%%%%%%%
%% Biblio
%%%%%%%%%%%%%%%%%%%%

\bibliographystyle{apalike}
\bibliography{biblio}


\end{document}
